% Options for packages loaded elsewhere
\PassOptionsToPackage{unicode}{hyperref}
\PassOptionsToPackage{hyphens}{url}
%
\documentclass[
  ignorenonframetext,
  aspectratio=169]{beamer}
\title{Correlation}
\subtitle{Definition, types of correlation, scatter diagram, Karl
Pearson's coefficient of correlation (linear correlation)}
\author{Deependra Dhakal}
\date{}
\institute{Assistant Professor \and Agriculture and Forestry
University \and \url{https://rookie.rbind.io}}

\usepackage{pgfpages}
\setbeamertemplate{caption}[numbered]
\setbeamertemplate{caption label separator}{: }
\setbeamercolor{caption name}{fg=normal text.fg}
\beamertemplatenavigationsymbolsempty
% Prevent slide breaks in the middle of a paragraph
\widowpenalties 1 10000
\raggedbottom
\setbeamertemplate{part page}{
  \centering
  \begin{beamercolorbox}[sep=16pt,center]{part title}
    \usebeamerfont{part title}\insertpart\par
  \end{beamercolorbox}
}
\setbeamertemplate{section page}{
  \centering
  \begin{beamercolorbox}[sep=12pt,center]{part title}
    \usebeamerfont{section title}\insertsection\par
  \end{beamercolorbox}
}
\setbeamertemplate{subsection page}{
  \centering
  \begin{beamercolorbox}[sep=8pt,center]{part title}
    \usebeamerfont{subsection title}\insertsubsection\par
  \end{beamercolorbox}
}
\AtBeginPart{
  \frame{\partpage}
}
\AtBeginSection{
  \ifbibliography
  \else
    \frame{\sectionpage}
  \fi
}
\AtBeginSubsection{
  \frame{\subsectionpage}
}
\usepackage{amsmath,amssymb}
\usepackage{lmodern}
\usepackage{iftex}
\ifPDFTeX
  \usepackage[T1]{fontenc}
  \usepackage[utf8]{inputenc}
  \usepackage{textcomp} % provide euro and other symbols
\else % if luatex or xetex
  \usepackage{unicode-math}
  \defaultfontfeatures{Scale=MatchLowercase}
  \defaultfontfeatures[\rmfamily]{Ligatures=TeX,Scale=1}
\fi
\usetheme[]{Frankfurt}
\usecolortheme{beaver}
% Use upquote if available, for straight quotes in verbatim environments
\IfFileExists{upquote.sty}{\usepackage{upquote}}{}
\IfFileExists{microtype.sty}{% use microtype if available
  \usepackage[]{microtype}
  \UseMicrotypeSet[protrusion]{basicmath} % disable protrusion for tt fonts
}{}
\makeatletter
\@ifundefined{KOMAClassName}{% if non-KOMA class
  \IfFileExists{parskip.sty}{%
    \usepackage{parskip}
  }{% else
    \setlength{\parindent}{0pt}
    \setlength{\parskip}{6pt plus 2pt minus 1pt}}
}{% if KOMA class
  \KOMAoptions{parskip=half}}
\makeatother
\usepackage{xcolor}
\IfFileExists{xurl.sty}{\usepackage{xurl}}{} % add URL line breaks if available
\IfFileExists{bookmark.sty}{\usepackage{bookmark}}{\usepackage{hyperref}}
\hypersetup{
  pdftitle={Correlation},
  pdfauthor={Deependra Dhakal},
  hidelinks,
  pdfcreator={LaTeX via pandoc}}
\urlstyle{same} % disable monospaced font for URLs
\newif\ifbibliography
\setlength{\emergencystretch}{3em} % prevent overfull lines
\providecommand{\tightlist}{%
  \setlength{\itemsep}{0pt}\setlength{\parskip}{0pt}}
\setcounter{secnumdepth}{-\maxdimen} % remove section numbering
\usepackage{setspace}
\usepackage{wasysym}
\usepackage{fontenc}
\usepackage{booktabs,siunitx}
\usepackage{longtable}
\usepackage{array}
\usepackage{multirow}
\usepackage{wrapfig}
\usepackage{float}
\usepackage{colortbl}
\usepackage{pdflscape}
\usepackage{tabu}
\usepackage{threeparttable}
\usepackage{threeparttablex}
\usepackage[normalem]{ulem}
\usepackage{makecell}
\usepackage{xcolor}
\usepackage{tikz} % required for image opacity change
\usepackage[absolute,overlay]{textpos} % for text formatting
\usepackage[skip=0.3\baselineskip]{caption}
% \usepackage{newtxtext,newtxmath}% better than txfonts   

% this font option is amenable for beamer
\setbeamerfont{caption}{size=\tiny}
\singlespacing

\sisetup{per-mode=symbol}

\newcommand\Myperm[2][^n]{\prescript{#1\mkern-2.5mu}{}P_{#2}}
\newcommand\Mycomb[2][^n]{\prescript{#1\mkern-0.5mu}{}C_{#2}}

% \setcounter{totalnumber}{50}
% \setcounter{topnumber}{50}
% \setcounter{bottomnumber}{50}
% \setlist[itemize,1]{leftmargin=2pt,itemindent=2pt}
% \setlist[itemize,2]{leftmargin=6pt,itemindent=2pt}
% \def\labelitemi{$\bullet$}
% \def\labelitemii{$\diamond$}
% \def\labelitemiii{\textbullet}
\captionsetup[table]{skip=0pt}
\captionsetup[table]{belowskip=-2pt}
\setbeamerfont{caption}{size=\tiny}
\setbeamertemplate{footline}[page number]
\AtBeginSection{}
\ifLuaTeX
  \usepackage{selnolig}  % disable illegal ligatures
\fi

\begin{document}
\frame{\titlepage}

\begin{frame}[allowframebreaks]
  \tableofcontents[hideallsubsections]
\end{frame}
\hypertarget{correlation}{%
\section{Correlation}\label{correlation}}

\begin{frame}{Meaning and definition}
\protect\hypertarget{meaning-and-definition}{}
\begin{itemize}
\tightlist
\item
  Suppose we have a sample of n pairs for which each pair represents the
  measurement of two variables, \(X\) and \(Y\). If a scatterplot of
  \(Y\) versus \(X\) shows a general linear trend, then it is natural to
  try to describe the strength of the linear association.
\item
  The systematic interrelationship between the two continuous related
  variables say, X and Y is termed as correlation. When only two
  variables are involved, the correlation is called simple correlation.
  If more than two variables are involved, the correlation is said to be
  multiple correlation.
\item
  When the variables move in the same direction, i.e., increase in one
  variable causes and increase in other variable and \emph{vice versa},
  such type of correlation is called positive/direct correlation. In
  general, grain yield of wheat and the number of grains per spike are
  positively correlated.
\item
  By analogy, negative correlation is said to occur when increase in one
  variable is followed by decrease in other. For example, grain yield of
  wheat and severity of disease in the field are negatively correlated.
\end{itemize}
\end{frame}

\begin{frame}{}
\protect\hypertarget{section}{}
\begin{columns}[T, onlytextwidth]
\column{0.5\textwidth}
\small Anscombe's Quartet

\begin{table}

\caption{\label{tab:anscombe_quartet}Anscombes quartet is a set of 4 ($x,y$) data sets that were published by Francis Anscombe in a 1973 paper Graphs in statistical analysis.}
\centering
\fontsize{5}{7}\selectfont
\begin{tabular}[t]{>{\raggedleft\arraybackslash}p{1.5em}>{\raggedleft\arraybackslash}p{1.5em}>{\raggedleft\arraybackslash}p{1.5em}>{\raggedleft\arraybackslash}p{1.5em}>{\raggedleft\arraybackslash}p{2em}>{\raggedleft\arraybackslash}p{2em}>{\raggedleft\arraybackslash}p{2em}>{\raggedleft\arraybackslash}p{2em}r}
\toprule
SN & x1 & x2 & x3 & x4 & y1 & y2 & y3 & y4\\
\midrule
\cellcolor{gray!6}{1} & \cellcolor{gray!6}{10} & \cellcolor{gray!6}{10} & \cellcolor{gray!6}{10} & \cellcolor{gray!6}{8} & \cellcolor{gray!6}{8.04} & \cellcolor{gray!6}{9.14} & \cellcolor{gray!6}{7.46} & \cellcolor{gray!6}{6.58}\\
2 & 8 & 8 & 8 & 8 & 6.95 & 8.14 & 6.77 & 5.76\\
\cellcolor{gray!6}{3} & \cellcolor{gray!6}{13} & \cellcolor{gray!6}{13} & \cellcolor{gray!6}{13} & \cellcolor{gray!6}{8} & \cellcolor{gray!6}{7.58} & \cellcolor{gray!6}{8.74} & \cellcolor{gray!6}{12.74} & \cellcolor{gray!6}{7.71}\\
4 & 9 & 9 & 9 & 8 & 8.81 & 8.77 & 7.11 & 8.84\\
\cellcolor{gray!6}{5} & \cellcolor{gray!6}{11} & \cellcolor{gray!6}{11} & \cellcolor{gray!6}{11} & \cellcolor{gray!6}{8} & \cellcolor{gray!6}{8.33} & \cellcolor{gray!6}{9.26} & \cellcolor{gray!6}{7.81} & \cellcolor{gray!6}{8.47}\\
6 & 14 & 14 & 14 & 8 & 9.96 & 8.10 & 8.84 & 7.04\\
\cellcolor{gray!6}{7} & \cellcolor{gray!6}{6} & \cellcolor{gray!6}{6} & \cellcolor{gray!6}{6} & \cellcolor{gray!6}{8} & \cellcolor{gray!6}{7.24} & \cellcolor{gray!6}{6.13} & \cellcolor{gray!6}{6.08} & \cellcolor{gray!6}{5.25}\\
8 & 4 & 4 & 4 & 19 & 4.26 & 3.10 & 5.39 & 12.50\\
\cellcolor{gray!6}{9} & \cellcolor{gray!6}{12} & \cellcolor{gray!6}{12} & \cellcolor{gray!6}{12} & \cellcolor{gray!6}{8} & \cellcolor{gray!6}{10.84} & \cellcolor{gray!6}{9.13} & \cellcolor{gray!6}{8.15} & \cellcolor{gray!6}{5.56}\\
10 & 7 & 7 & 7 & 8 & 4.82 & 7.26 & 6.42 & 7.91\\
\cellcolor{gray!6}{11} & \cellcolor{gray!6}{5} & \cellcolor{gray!6}{5} & \cellcolor{gray!6}{5} & \cellcolor{gray!6}{8} & \cellcolor{gray!6}{5.68} & \cellcolor{gray!6}{4.74} & \cellcolor{gray!6}{5.73} & \cellcolor{gray!6}{6.89}\\
\bottomrule
\end{tabular}
\end{table}

\column{0.5\textwidth}
\small Rothamsted Oats

\begin{table}

\caption{\label{tab:rothamsted_oats}A dataset from RCB experiment, carried out at Rothamsted facility, of oats taking measurements on straw and grain with 9 fertilizer treatments.}
\centering
\fontsize{5}{7}\selectfont
\begin{tabular}[t]{>{\raggedleft\arraybackslash}p{5em}>{\raggedleft\arraybackslash}p{5em}l}
\toprule
SN & grain & straw\\
\midrule
\cellcolor{gray!6}{1} & \cellcolor{gray!6}{61.375} & \cellcolor{gray!6}{83}\\
2 & 68.75 & 130\\
\cellcolor{gray!6}{3} & \cellcolor{gray!6}{64.25} & \cellcolor{gray!6}{100}\\
4 & 65.5 & 96\\
\cellcolor{gray!6}{5} & \cellcolor{gray!6}{79.625} & \cellcolor{gray!6}{130.5}\\
6 & 79.25 & 122\\
\cellcolor{gray!6}{7} & \cellcolor{gray!6}{$\dots$} & \cellcolor{gray!6}{$\dots$}\\
8 & $\dots$ & $\dots$\\
\cellcolor{gray!6}{9} & \cellcolor{gray!6}{$\dots$} & \cellcolor{gray!6}{$\dots$}\\
10 & 82.125 & 175.5\\
\cellcolor{gray!6}{11} & \cellcolor{gray!6}{83.75} & \cellcolor{gray!6}{140.5}\\
12 & 84.75 & 122\\
\cellcolor{gray!6}{13} & \cellcolor{gray!6}{83.875} & \cellcolor{gray!6}{192.5}\\
14 & 89 & 188\\
\cellcolor{gray!6}{15} & \cellcolor{gray!6}{93.25} & \cellcolor{gray!6}{162}\\
\bottomrule
\end{tabular}
\end{table}

\end{columns}

\footnotesize

\begin{itemize}
\tightlist
\item
  We take the two datasets (above) and show (graphically) following
  types of correlation in scatterplot:

  \begin{itemize}
  \scriptsize
  \item Positive correlation
  \item Negative correlation
  \item Dinosaur correlation
  \end{itemize}
\end{itemize}
\end{frame}

\begin{frame}{Scatterplot diagram}
\protect\hypertarget{scatterplot-diagram}{}
\begin{figure}

{\centering \includegraphics[width=0.98\linewidth]{08-correlation_files/figure-beamer/correlation-scatter-plot-1} 

}

\caption{Association between variables in two datasets (Anscombe's Quartet: Left; Rothamsted experiment: Right)}\label{fig:correlation-scatter-plot}
\end{figure}
\end{frame}

\begin{frame}[fragile]{}
\protect\hypertarget{section-1}{}
\begin{verbatim}
## [1] 9.00000000000000 9.00000000000000 9.00000000000000 9.00000000000000
## [5] 7.50090909090909 7.50090909090909 7.50000000000000 7.50090909090909
\end{verbatim}

\begin{verbatim}
## [1] 11.00000000000000 11.00000000000000 11.00000000000000 11.00000000000000
## [5]  4.12726909090909  4.12762909090909  4.12262000000000  4.12324909090909
\end{verbatim}

\begin{verbatim}
## [1] 0.816420516344840 0.816236506000243 0.816286739489598 0.816521436888503
\end{verbatim}
\end{frame}

\begin{frame}{}
\protect\hypertarget{section-2}{}
\begin{tabular}{llrrrrrrrrrl}
\toprule
skim\_type & skim\_variable & n\_missing & complete\_rate & numeric.mean & numeric.sd & numeric.p0 & numeric.p25 & numeric.p50 & numeric.p75 & numeric.p100 & numeric.hist\\
\midrule
numeric & x1 & 0 & 1 & 9.00000000000000 & 3.31662479035540 & 4.00 & 6.500 & 9.00 & 11.50 & 14.00 & <U+2587><U+2585><U+2585><U+2585><U+2585>\\
numeric & x2 & 0 & 1 & 9.00000000000000 & 3.31662479035540 & 4.00 & 6.500 & 9.00 & 11.50 & 14.00 & <U+2587><U+2585><U+2585><U+2585><U+2585>\\
numeric & x3 & 0 & 1 & 9.00000000000000 & 3.31662479035540 & 4.00 & 6.500 & 9.00 & 11.50 & 14.00 & <U+2587><U+2585><U+2585><U+2585><U+2585>\\
numeric & x4 & 0 & 1 & 9.00000000000000 & 3.31662479035540 & 8.00 & 8.000 & 8.00 & 8.00 & 19.00 & <U+2587><U+2581><U+2581><U+2581><U+2581>\\
numeric & y1 & 0 & 1 & 7.50090909090909 & 2.03156813592582 & 4.26 & 6.315 & 7.58 & 8.57 & 10.84 & <U+2583><U+2582><U+2587><U+2583><U+2583>\\
\addlinespace
numeric & y2 & 0 & 1 & 7.50090909090909 & 2.03165673550162 & 3.10 & 6.695 & 8.14 & 8.95 & 9.26 & <U+2581><U+2581><U+2581><U+2581><U+2587>\\
numeric & y3 & 0 & 1 & 7.50000000000000 & 2.03042360112367 & 5.39 & 6.250 & 7.11 & 7.98 & 12.74 & <U+2587><U+2586><U+2582><U+2581><U+2582>\\
numeric & y4 & 0 & 1 & 7.50090909090909 & 2.03057851138760 & 5.25 & 6.170 & 7.04 & 8.19 & 12.50 & <U+2587><U+2587><U+2583><U+2581><U+2582>\\
\bottomrule
\end{tabular}
\end{frame}

\begin{frame}{Mathematical representation of correlation}
\protect\hypertarget{mathematical-representation-of-correlation}{}
\end{frame}

\begin{frame}{Karl pearson's coefficient of correlation (linear
correlation)}
\protect\hypertarget{karl-pearsons-coefficient-of-correlation-linear-correlation}{}
\end{frame}

\begin{frame}{Correlation coefficient for bivariate frequency
distribution}
\protect\hypertarget{correlation-coefficient-for-bivariate-frequency-distribution}{}
\end{frame}

\end{document}
